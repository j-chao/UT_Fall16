\documentclass{article}
% packages
\usepackage {lmodern}
\usepackage [T1]{fontenc}
\usepackage {amsmath}
\usepackage {amssymb}
\usepackage {amsfonts}
\usepackage {graphicx}
\usepackage {fullpage}
\usepackage {gensymb}
\usepackage {caption}
\usepackage {subcaption}
\usepackage {array}
\newcolumntype{L}{>{\centering\arraybackslash}m{3cm}}
%\usepackage{nopageno}
\usepackage {cite}
\usepackage {setspace}
\usepackage [version=4]{mhchem}
\usepackage {pdfpages}
\usepackage{fancyvrb}

\graphicspath {{images/}}

\begin{document}

\title{An Investigation on the Efficiency of Multiple Sequence Alignment Tools}
\author {Justin Chao - juchao \\
		Chase Meyer - cmeyer3 \\
		Bria Lacour - lacour}

\maketitle

\section*{Proposal}
We propose the development of an alignment program that uses both
global and local alignment algorithms to create a Multiple Sequence Alignment
(MSA) tool that is both accurate and efficient. 

Our efforts will involve the usage of random walk algorithms for generating a
BLOck SUbstitution Matrix (BLOSUM). The optimal sequence alignment will then be
calculated using Needleman-Wunsch and Smith-Waterman algorithms, and a
comparison of efficiency will be conducted between the two algorithms. A
comparison of run-times will then be conducted between our developed MSA tool,
BLAST, and MAFFT on a series of test sequences. 

\section*{Background}
A MSA tool can be used by computational biologists to track evolutionary
relationships, predict structures, track mutations, etc. MSA tools come in a
variety of forms and are tailored to address specific biological problems.
These sequences are usually protein, DNA, or RNA, which can affect the methods
of analysis in different ways.  

BLAST, the most widely used MSA algorithm, works by generating a matrix of
probability values called a BLOSUM matrix. It calculates the probability of
certain regions in one sequence aligning with a number of other reference
sequences, and generating matrices to store the probability values. This is
referred to as scoring and is done using amino-acid similarity in the
sequences. The final output from BLAST is the sequence alignment corresponding
to the highest probability values determined from these calculated matrices. 

The efficiency of these tools can affect scientific analyses, such as
identifying significant patterns in protein families or assessing conservation
of different genetic characteristics across species.  BLAST, and other commonly
used alignment tools, involve matrix calculations using the Needleman-Wunsch
algorithm or the Smith-Waterman algorithm. While the Needleman-Wunsch algorithm
is slightly more accurate, the Smith-Waterman algorithm is used more often for
its greater efficiency, a necessity when performing calculations on large
matrix sizes. 

\section*{Tools}
\begin{itemize}
    \item BLAS (Matrix data storage, manipulation, and calculations)
    \item Probability and Statistical Methods
    \item Timers and Profilers
\end{itemize}


\section*{Algorithms}
\subsection*{Needleman-Wunsch Algorithm}
\begin{align*}
    M(0, j) = j \times p  \hspace{1cm}&\text{for first row, where p is the gap penaly} \\
    M(i, 0) = i \times p  \hspace{1cm}&\text{for first column}
\end{align*}

\begin{equation*}
    M(0, j) = max \begin{cases}
        M(i-1, j) + p & \text{top} \\
        M(i, j-1) + p & \text{left} \\
        M(i-1, j-1) + s(a_j , b_i) & \text{diagonal}
  \end{cases}
\end{equation*}

Where $s(a_j , b_i)$ = match/mismatch score for sites j and i in sequences a and b.

\subsection*{Smith-Waterman Algorithm}
\begin{align*}
    M(0, j) = j \times p  \hspace{1cm}&\text{for first row} \\
    M(i, 0) = i \times p  \hspace{1cm}&\text{for first column}
\end{align*}

\begin{equation*}
    M(0, j) = max \begin{cases}
        0 \\
        M(i-1, j) + p & \text{top} \\
        M(i, j-1) + p & \text{left} \\
        M(i-1, j-1) + s(a_j , b_i) & \text{diagonal}
  \end{cases}
\end{equation*}

Where $s(a_j , b_i)$ = match/mismatch score for sites j and i in sequences a and b.

\section*{Research Significance}
The goal of BLAST is to find the best alignment, given a
scoring system that varies in different multiple sequence alignment software.
What makes this project interesting is the fact that there are so many ways to
score these alignments and that a potentially significant alternative could be
proposed. We would also like to compare the findings of our software to BLAST
and other multiple sequence alignment methods. The goal is to develop and
optimize or even propose a new way to align sequences that involves algorithms
different than the standard to score the sequence alignments and generate
substitution matrices. We would even like to attempt to implement parallelism,
if time permits, which would add an interesting aspect in comparing efficiency
and optimization.

\section*{References}
\begin{itemize}
	\item Multiple Sequence Alignment,                      https://goo.gl/2ulXvM
	\item Fast Fourier Transform Analysis of DNA Sequences, https://goo.gl/bCNHMC
	\item The Wilke Lab,                                    https://goo.gl/QshFXK
    \item Department of Computer Science, Columbia University, https://goo.gl/tli4St
\end{itemize}

\end{document}
